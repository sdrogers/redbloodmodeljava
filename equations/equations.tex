\documentclass[a4paper]{article}
\usepackage{url}
\usepackage[margin=2cm]{geometry}
\usepackage{amsmath}
\title{governing-equations-of-the-rcm}

% \newcommand{\med}[1]{[#1]^m}
\newcommand{\med}[1]{M#1}
% \newcommand{\cell}[1]{[#1]^c}
\newcommand{\cell}[1]{C#1}


\newcommand{\MA}{\med{A}}
\newcommand{\MB}{\med{B}}
\newcommand{\MBH}{\med{BH}}
\newcommand{\MGluca}{M_{glucamine}}
\newcommand{\MNa}{\med{Na}}
\newcommand{\MK}{\med{K}}
\newcommand{\MCatp}{\med{Ca^{2+}}}
\newcommand{\MMgtp}{\med{Mg^{2+}}}
\newcommand{\MGluco}{M_{gluconate}}
\newcommand{\MH}{\med{H}}
\newcommand{\MOs}{\med{Os}}
\newcommand{\MCat}{\med{Cat}}
\newcommand{\MMgt}{\med{Mgt}}


\newcommand{\KB}{K_B}
\newcommand{\KBCa}{K_{{BC}_a}}
\newcommand{\KCa}{K_{Ca}}

\newcommand{\Km}[1]{Km_{#1}}
\newcommand{\KI}[1]{KI_{#1}}

\newcommand{\CNa}{\cell{Na}}
\newcommand{\CK}{\cell{K}}
\newcommand{\CH}{\cell{H}}
\newcommand{\CMgtp}{\cell{Mg^{2+}}}
\newcommand{\CCatp}{\cell{Ca^{2+}}}
\newcommand{\CA}{\cell{A}}
\newcommand{\nHb}{n_{Hb}}
\newcommand{\CHb}{\cell{Hb}}
\newcommand{\nX}{n_{X}}
\newcommand{\CXm}{\cell{X^{-}}}
\newcommand{\COs}{\cell{Os}}
\newcommand{\CCa}{\cell{Ca}}
\newcommand{\CCaB}{\cell{CaB}}
\newcommand{\CBCa}{\cell{B_{Ca}}}
\newcommand{\CMgB}{\cell{MgB}}
\newcommand{\CMg}{\cell{Mg}}
\newcommand{\CX}{\cell{X}}

\newcommand{\QX}{QX}
\newcommand{\QCa}{QCa}

\newcommand{\pH}[1]{pH_{#1}}
\newcommand{\pI}{pI}

\newcommand{\fHb}{f_{Hb}}


\newcommand{\F}[2]{F_{#1}#2}

\newcommand{\kk}[2]{k_{#1}#2}

\begin{document}
\maketitle

\section{Introduction}\label{introduction}

Red blood cell homeostasis addresses the subset of mechanisms that
control the dynamic changes in cell volume, membrane potential, ionic
composition, membrane transport and osmotic gradients in response to
perturbations. The modelled system consists of a suspension of identical
RBCs whose dynamic behaviour is constrained only by charge and mass
conservation. The equations implement these laws following a strict
computational sequence representative of the multiple interconnected
processes involved, delivering true and tested predictions on the
homeostatic behaviour of human RBCs in physiological, pathological and
experimental conditions.

We retain in the equations the names of parameters and variables as used for model inputs and outputs where the use of conventional nomenclatures for concentrations (i.e. [Ca2+]c, [Mg2+]o) or of Greek symbols for fluxes was not feasible.  Coherence with model operation in the choice of equation nomenclature was considered more important than compliance with established nomenclature, clarity of meaning being preserved in the unconventional style and fully explained in the Appendix glossaries of the User Guide (\url{https://github.com/sdrogers/redcellmodeljava}).


\subsection{The initial Reference State(RS)}

The RBC reference state describes the initial condition of the system in
a pump-leak balanced steady state. For compliance with initial
electroneutrality and osmotic equilibrium we use a phenomenology in
which the charge, $\nX$, and cell content of the global, non-haemoglobin,
impermeant cell anion, $\QX$ , are treated as wildcard parameters in
equations 3 and 8. The $\nX$ and $\CXm$ values emerging from such treatment
correspond closely with the known organic and inorganic phosphate pools
of metabolically normal RBCs {[}3{]}. When modifying the initial default
values in the RS the wildcard parameters may change. The model
automatically recalculates their value potentially changing slightly the
constitutive make up of the impermeant cell anion ($\nX \times \QX$) in
the new cell.

\subsubsection{Mediumelectroneutrality}\label{medium-electroneutrality}

\begin{equation}
\MA + (\MB - \MBH) + \MGluco-(\MNa + \MK + 2(\MCatp + \MMgtp + \MGluca)) = 0 \tag{1}
\end{equation}

Medium concentration of proton-bound buffer, $\MBH$ (HEPES, by default):

\begin{equation}
\MBH = \MB\left(\frac{\MH}{\KB + \MH}\right)
\end{equation}

\subsubsection{Intracellular electroneutrality}

\begin{equation}
\CNa + \CK + \CH + 2\CMgtp + 2\CCatp - (\CA + \nHb\CHb + \nX\CXm) = 0
\end{equation}

$\nHb$, the net charge on the haemoglobin molecule, is represented by the Cass-Dalmark equation [4],

\begin{equation}
\nHb=  a(\pH{i} - \pI)
\end{equation}

where $a$ corresponds to the linear segment of the proton titration curve of Hb in intact RBCs, and $\pI$ is the $\pH{i}$ at the isoelectric point of haemoglobin.

\subsubsection{Medium and cell osmolarities, MOs and COs:}

\begin{eqnarray}
	\MOs &=& \MA + \MB + \MGluco + \MNa + \MK + \MCat + \MMgt + \MGluca\\
	\COs &=& \CNa + \CK + \CA + \CH + \CMgtp + \CCatp  + \fHb\CHb + \CXm
\end{eqnarray}

$\fHb$ is the osmotic coefficient of haemoglobin, represented with only two virial coefficients, $b$ and $c$:

\begin{equation}
\fHb = 1 + b\times\CHb + c\times\CHb^2 
\end{equation}

\subsubsection{Osmotic equilibrium in the reference steady state:}

\begin{equation}
\MOs = \COs
\end{equation}

\subsubsection{Cytoplasmic buffering of protons, calcium and magnesium.}
Heamoglobin is the major cytoplasmic buffer for protons (eq 4) and for calcium ($\alpha$-buffer in eq 9c).  The main magnesium buffers are ATP and 2,3-DPG, compounds integrated within the X  phenomenology.  Because the bound forms of Ca and Mg are contained within CX , they are not included as separate osmolarity contributors in eq 6, leaving only the free forms of $Ca^{2+}$ and $Mg^{2+}$ as osmotic contributors.  

Cytoplamic $Ca^{2+}$ and $Mg^{2+}$ buffering have been measured with precision in intact RBCs [5-8] enabling accurate representations in the model.  The total Ca and Mg content of the cells, QCa and QMg, is reported in units of mmol/(340g Hb) (or mmol/Loc) whereas concentrations of the free forms, $\CCatp$ and $\CMgtp$, are expressed in units of mmol/Lcw, a conversion requiring translation for operational reasons in the model   Equation 9a translates QCa in units of mmol/Loc to CCa in units of mmol/Lcw using:


\setcounter{equation}{0}
\renewcommand{\theequation}{9.\alph{equation}}

\begin{equation}
\CCa = \QCa\frac{RCV}{Vw}  
\end{equation}
 

The total calcium concentration is the sum of free and bound forms:

\begin{equation}
\CCa = \CCatp + \CCaB
\end{equation}

There are two buffer systems for binding calcium in the RBC cytoplasm, $\alpha$ (mostly haemoglobin), and the BCa/KBCa buffer [6]. The concentration of bound calcium, CCaB, at each total calcium concentration, CCa, is represented by:

\begin{equation}
\CCaB = \alpha\CCa + \CBCa\frac{\CCatp}{\CCatp + \KBCa}
\end{equation}

CCa2+ is solved from the implicit equation: 

\begin{equation}
\CCa – \CCatp – \CCaB = 0  
\end{equation}

by the Newton-Raphson routine in the RS and at the end of the computations in each iteration cycle.  The measured values of the calcium binding parameters are $\alpha = 0.30$, $\CBCa = 0.026$ mmol/Loc, and $\KBCa = 0.014$ mM [6].

The corresponding equations for cytoplasmic magnesium buffering and $CMg^{2+}$ are:

\begin{eqnarray}
\CMg &=& QMg\frac{RCV}{Vw}\\
\CMg &=& \CMgtp + \CMgB\\
\CMgB &=& CBMg1\frac{\CMgtp}{\CMgtp+KBMg1}+CBMg2\frac{\CMgtp}{\CMgtp+KBMg2}+CBMg3
\end{eqnarray}

\emph{is $KBMG2$ actually $K \times BMg1$}

$\CMgtp$ is solved from the implicit equation: 

\begin{equation}
\CMg - \CMgtp - \CMgB = 0
\end{equation}


The measured values of the Mg bufferes [8] are: CBMg1 = 1.2 mmol/Loc, KBMg1 = 0.08 mM; CBMg2 = 7.5 mmol/Loc (15 mEq/Loc), KBMg2 = 3.6 mM; CBMg3 = 0.05 mmol/Loc. BMg1 represents ATP, BMg2 represents 2,3-DPG and miscellaneous phosphate groups, and BMg3 is an unidentified high affinity magnesium buffer.  



\subsubsection{Effects of deoxygenation on cytoplasmic Mg2+ buffering and pHi.}
Deoxygenation increases haemoglobin binding of ATP and 2,3-DPG thus reducing their availability for buffering intracellular magnesium.  CBMg1 is reduced by half and CBMg2 by 1.7 [8].  This is particularly relevant for simulating accurately the effects of changing the oxygenation condition of RBCs, a process in which changes in CMg2+ become enmeshed with effects arising from changes in the isoelectric point of haemoglobin, pI (eq 4).  In vivo, RBC are continuously changing between oxy and deoxy states as they flow between the arterial and venous vasculature causing well documented alternating changes in pHi, CMg2+, CA and cell volume which the model accurately reproduces.  Although these changes are fully reversible in physiological conditions, deoxygenation of sickle RBCs can lead to hyperdense collapse, as shown in the next paper [2]. 

Hb is assumed to be in a oxy-state by default, the most frequent experimental condition. Deoxygenation of Hb (Deoxy) changes its pI(0oC) from 7.2 to 7.5.  The model automatically adjusts the actual pI change for the temperature of the experiment.  The pI shifts during oxy-deoxy transitions cause sudden changes in the protonization condition of Hb with secondary changes in pHi and CMg2+, changes which the model predicts with verified accuracy [9-11].  Electroneutrality preservation during oxy-deoxy transitions requires constancy of nHb values (eq 4) when pI changes, from which the compensatory changes in pHi can be derived according to [10]: 

On deoxygenation:

\setcounter{equation}{0}
\renewcommand{\theequation}{4.\alph{equation}}
\begin{equation}
pHideoxy = pHioxy + pIdeoxy - pIoxy 
\end{equation}
On reoxygenation:
\begin{equation}
pHioxy = pHideoxy + pIoxy - pIdeoxy 
\end{equation}

\subsection{The dynamic state}

\setcounter{equation}{9}
\renewcommand{\theequation}{\arabic{equation}}

A first requirement at the start of simulations is to define the relative volume occupied by cells in the cell suspension system, the cell volume fraction, CVF.  Perturbations alter the flux of transported solutes and water across the plasma membrane of the cell thus initiating a cascade of downstream changes in the compositions of cell and suspending medium.  It is therefore important to start by listing the membrane transport component of the cell and of the equations describing their basic kinetic properties.  

\subsubsection{Flux equations of the model, Fi}
The substrates of the RBC membrane transporters are Na, K, A, H, Ca, Mg and water, the "i" in Fi.  The sign-convention applied in the equations is for positive net fluxes into the cell (influx) and for negative net fluxes into the medium (efflux). The name convention adopted here for the transport of substrate X by the different membrane transporters is as follows: FPX = pump-mediated flux of X, with P = NaP for the Na/K pump or CaP for the calcium pump (PMCA); FGX = X-flux through electrodiffusional channel defined with constant field kinetics; FXA = electroneutral carrier-mediated cotransport of cation X and anion A defined with low-saturation kinetics; FzX = electrodiffusional flux of X through PIEZO1 channel; FCoX = electroneutral cotransport of X mediated by the Na:K:2Cl symport, of minimal expression and activity in human RBCs; FA23X = electroneutral M2+:2H+ exchange flux through the divalent cation ionophore A23187, the only exogenous membrane transporter included in the model; Fw = water flux mediated mainly by aquaporins and partly by partition diffusion through the plasma membrane.  

\setcounter{equation}{0}
\renewcommand{\theequation}{10.\alph{equation}}

\subsubsection{Flux pathways for each transported substrate:}
\begin{eqnarray}
\F{}{Na} &=& \F{NaP}{Na} + \F{G}{Na} + \F{Na}{A} + \F{Co}{Na} + \F{z}{Na}\\
\F{K} &=& \F{NaP}{K} + \F{G}{K} + \F{K}{A} + \F{KGardos} + \F{Co}{K} + \F{z}{K}\\
\F{}{A} &=& \F{G}{A} + \F{H}{A} + \F{Na}{A} + \F{K}{A} + \F{z}{A} + 2\times\F{Co}{A}\\
\F{}{H} &=& \F{G}{H} + \F{H}{A} + \F{Ca}{PH} + \F{A23}{H} \\
\F{}{Ca} &=& \F{CaP}{Ca} + \F{G}{Ca} + \F{z}{Ca} + \F{A23}{Ca} \\
\F{}{Mg} &=& \F{A23}{Mg} \\
\F{}{w}  &=& Pw*(\COs – \MOs)
\end{eqnarray}

There are no data on PIEZO1-mediated Mg2+ fluxes in RBCs.  Although PzMg most certainly has a small finite value, $\F{z}{Mg}$ is likely to be very small under the usually low electrochemical $Mg^{2+}$ gradients across the RBC membrane.  With this level of ignorance and uncertainty, $\F{z}{Mg}$ was not included in the current model version.  

\subsubsection{Kinetic descriptions of individual transporters}
Certain transporter kinetics are reported in the equations with the default numerical values used for dissociation and rate constants in the model, based on well established values in the literature and on the good semi-quantitative fits to experimental data provided in the past [12-14].  

\setcounter{equation}{0}
\renewcommand{\theequation}{11.\alph{equation}}


\subsubsection{Na/K pump mediated fluxes of Na and K (f = forward; r = reverse) [15, 16]}

\begin{eqnarray}
\F{NaP}{Na}^f  &=& -\F{NaP}{max}^f\times\left(\frac{\CNa}{\CNa + 0.2(1 + \CK/8.3)}\right)^3 \times\left(\frac{\MK}{\MK + 0.1(1 + \MNa/18))}\right)^2 \\
\F{NaP}{Na}^r  &=& \F{NaP}{max}^r\times\left(\frac{\CK}{(\CK + 8.3(1 + \CNa/0.2))}\right)^2\times\left(\frac{\CNa}{\CNa + 18(1 + \CK/0.1))}\right)^3 \\
\F{NaP}{Na} &=& \F{NaP}{Naf} + \F{NaP}{Nar} \\
\F{NaP}{K} &=& -\F{NaP}{Na}/1.5
\end{eqnarray}

\subsubsection{PMCA. Calcium and proton fluxes through the calcium pump operating as an electroneutral Ca:2H exchanger  [17, 18]}

\setcounter{equation}{0}
\renewcommand{\theequation}{12.\alph{equation}}

\begin{eqnarray}
\F{CaP}{Ca} &=& -\kk{CaP}{}\times\frac{(\CCatp)^4}{(0.0002)^4 + (\CCa2+)^4}\\
\F{CaP}{H} &=& -2*\F{CaP}{Ca}
\end{eqnarray}

\subsubsection{Electrodiffusional fluxes of i (Na, K, Ca, H and A) through endogenous channels, FGi , Gardos channels, FGGardos, and PIEZO1 channels, Fzi, are represented with constant field kinetics [19]: }

\setcounter{equation}{12}
\renewcommand{\theequation}{\arabic{equation}}
\begin{equation}
\F{G}{i} = -P_{G}i\times\frac{ziFEm}{RT}\times\frac{(Ci – Mi\exp^{-ziFEm/RT})}{(1 – exp^{-ziFEm/RT})}
\end{equation}
with PGi representing the Goldmanian i-permeability in h-1 units

\subsubsection{PGKGardos is a function of CCa2+ [20, 21] as follows:}
\begin{equation}
P_{G}K_{Gardos} = PK_{GardosMax}\times\frac{(\CCatp)^4}{(\KCa)^4 + (\CCatp)^4}
\end{equation}

\subsubsection{PGCa is a function of CCa2+ and MCa2+ [21, 22] as follows:}
\begin{equation}
P_{G}Ca = \left(\frac{\CCatp}{0.0002 + \CCatp}\right)\left(\frac{\MCatp}{0.8 + \MCatp}\right)
\end{equation}

\subsubsection{Low-saturation, carrier mediated flux phenomenology for electroneutral cotransporters FNaA, FKA and FHA.}

\setcounter{equation}{0}
\renewcommand{\theequation}{16.\alph{equation}}
\begin{eqnarray}
\F{Na}{A} &=& - \kk{NaA}{}(\CNa\times\CA – \MNa\times\MA) \\
\F{K}{A} &=& - \kk{KA}{}(\CK\times\CA – \MK\times\MA) \\
\F{H}{A} &=& - \kk{HA}{}(\CH\times\CA – \MH\times\MA) 
\end{eqnarray}

Note that $\kk{HA}{}$, the rate constant of the H:A cotransport phenomenology representing the operation of the Jacob-Stewart mechanism (JS) is between five and six orders of magnitude faster than that of any of the other ion transporters in the membrane (see User Guide for details and references).  

\setcounter{equation}{0}
\renewcommand{\theequation}{17.\alph{equation}}

\subsubsection{Electroneutral Na:K:2A cotransport}

\begin{eqnarray}
\F{Co}{} &=& -\kk{Co}{}((CNa\times\CK\times\CA^2) – d(MNa\times\MK\times\MA^2)) \\
d &=& \frac{\CNa\times\CK\times\CA^2}{\MNa\times\MK\times\MA^2}
\end{eqnarray}
The CX and MX values in eq 17b are those set for the RS 
\begin{eqnarray}
\F{Co}{Na} &=& \F{Co}{K} = \F{Co}{}\\
\F{Co}{A} &=& 2\F{Co}{}
\end{eqnarray}
$d$ is a wildcard factor introduced to set $\F{Co}{} = 0$ only in the RS. Its value is set by the initial Na, K and A concentrations in the RS. $d$ remains as a fixed-value parameter during dynamic state computations.    

\subsubsection{Electroneutral M2+:2H+ exchange fluxes of $Ca^{2+}$ and $Mg^{2+}$ mediated by the divalent cation ionophore A23187}

The divalent cation ionophore A23187 mediates an electroneutral $M^{2+}:2H^+$ exchange when incorporated into cell membranes [23]. Divalent cation ionophores became essential and extensively used tools in research on calcium and magnesium function and dysfunction in RBCs [5, 7, 24, 25] and in many other cell types. To emulate experimental protocols with the use of divalent cation ionophores it became necessary to represent their transport properties in the model as an optional exogenous transporter of the RBC membrane.  

In albumin-free RBC suspensions, the RBC/medium partition ratio of the lipophilic ionophore A23187 is 60/1, 20 to 50\% of it confined to the cell membrane [26].  The transport kinetics of the ionophore was modeled with symmetric binding (Km) and inhibitory (KI) dissociation constants for Ca2+ and Mg2+ on each membrane side, as follows:


\begin{eqnarray}
\nonumber A1 &=& \frac{\MCatp}{\Km{Ca}(1 + \MMgtp/\KI{Mg} + \MCatp)} \\
\nonumber A2 &=& \frac{\CCatp}{\Km{Ca}(1 + \CMgtp/\KI{Mg} + \CCatp)}  \\
\nonumber A3 &=& \frac{\MMgtp}{\Km{Mg}(1 + \MCatp/\KI{Ca} + \MMgtp)}  \\
\nonumber A4 &=& \frac{\CMgtp}{\Km{Mg}(1 + \CCatp/\KI{Ca} + \CMgtp)} 
\end{eqnarray}

Following extensive preliminary tests [27], default values of 10 mM for the four Km and KI parameter set were found to deliver excellent agreement between predicted and measured ionophore-mediated fluxes, and to ensure adequate compliance with the measured equilibrium distribution of the transported ions when ionophore-mediated net fluxes approach zero [23]: 
\[\CCatp/\MCatp \approx \CMgtp/\MMgtp \approx (\CH^+/\MH^+)^2.\]

Combining the Ca2+, Mg2+ and H+ driving gradients we obtain:
\begin{eqnarray}
\nonumber B1&=& A1(\CH)^2 - A2(\MH)^2 \\
\nonumber B2&=& A3(\CH)^2 - A4(\MH)^2
\end{eqnarray}

The ionophore-mediated fluxes of $Ca^{2+}$, $Mg^{2+}$ and $H^+$, $\F{A23}{Ca}$, $\F{A23}{Mg}$ and $\F{A23}{H}$, respectively, can now be computed from: 

\setcounter{equation}{0}
\renewcommand{\theequation}{A23-\arabic{equation}}

\begin{eqnarray}
\F{A23}{Ca} &=& P_{A23}B1 \\
\F{A23}{Mg} &=& P_{A23}B2 \\
\F{A23}{H} &=& -2(\F{A23}{Ca} + \F{A23}{Mg})
\end{eqnarray}

Where $P_{A23}$ is the ionophore-mediated permeability. $P_{A23}$ is a power function of the RBC ionophore concentration, $P_{A23} = 0.22*[I]1.45$, when $P_{A23}$ is expressed in units of 10-6 cm/s, and [I] in $\mu$mol/Loc [26, 28, 29]. Within the units-set in the model, numerical values of $P_{A23}$ in the range 1017 to 2*1018 offered a perfectly adequate minimalist emulation of the effects of different ionophore concentrations on the fluxes and distributions of $Ca^{2+}$, $Mg^{2+}$ and $H^{+}$ ions in RBCs in a large variety of experimental conditions [5, 27, 30-33].    

\subsubsection{Equation sequence for the computations of dynamic states.}
Following perturbations, sustained charge conservation and electroneutrality is implemented by:

\setcounter{equation}{0}
\renewcommand{\theequation}{18\alph{equation}}

\begin{equation}
\sum Ii = 0 
\end{equation}
where $Ii$ represents the current carried by each of the electrogenic transporters in the system. $\sum Ii$ is therefore the fist equation that has to be solved at the start of each iteration in the numerical computations of the model. Capacitative currents ($Ic = C(dV/dt))$ are ignored because their magnitude and time-course are orders of magnitude smaller than those of the homeostatic relevant currents.  The relation between currents and individual ion fluxes, $Fi$, is given by 

\begin{equation}
\sum Ii = F\sum zi*Fi 
\end{equation}
where $zi$ is the valence of ion $i$ and $F$ is the Faraday constant. 

With the electrogenic flux components in the model, $\sum ziFi = 0$ renders:

\begin{equation}
\sum ziFi = \F{NaP}{Na} + \F{NaP}{K} + \F{G}{Na} + \F{G}{K} + \F{G}{K_Gardos} + \F{G}{A} + \F{G}{Ca} + \F{G}{H} + \F{z}{Na} + \F{z}{K} + \F{z}{A} + \F{z}{Ca} = 0
\end{equation}
$\sum ziFi$ is a complex function of temperature, membrane potential, $Em$, and of the concentration of all transported and modulating substrates.  With all parameters, kinetics and substrate concentrations known $\sum Ii = 0$ becomes an implicit equation in $Em$, the single unknown left, solved in each iteration with the Newton-Raphson cord approximation routine.      

With $Em$, the new $Fi$ values for each of the electrodiffusional terms in eq 18c can be computed.  With the sum of the absolute values of all the $Fi$ we can now assign a new $\delta t$ duration to each iteration interval, as follows:  

\setcounter{equation}{18}
\renewcommand{\theequation}{\arabic{equation}}
\begin{equation}
\Delta t = \frac{a}{b + \sum|Fi|}
\end{equation}

The value of $a$, under user control, optimises $\Delta t$ scales for different simulations (“frequencyfactor” in the RCM); $b$ is a small zero-avoidance parameter in the denominator. The advantage of this strategy over using regular iteration intervals is that by setting a constant value for the cycles per outcome (“cyclesperprint(epochs)” in the RCM) the density of data output points automatically adjusts to the overall rate of change in the system, emulating the way good experimental practice seeks to sample for data at the bench, thus optimizing comparisons between predicted and experimental results.    

With the new $Fi(t)$ and $\Delta t$ the new $Qi(t)$ may be computed using the values of $FNa(t)$, $FK(t)$, $FA(t)$, $FH(t)$, $FCa(t)$ and $\F{A23}{Mg}(t)$ from equations (10a-f) as follows:

\setcounter{equation}{0}
\renewcommand{\theequation}{20\alph{equation}}

\begin{eqnarray}
\Delta QNa &=& FNa*\Delta t\\
\Delta QK &=& FK*\Delta t \\
\Delta QA &=& FA*\Delta t \\
\Delta H &=& FH*\Delta t \\
\Delta QCa &=& FCa*\Delta t \\
\Delta QMg &=& \F{A23}{Mg}*\Delta t \\
QNa(t) &=& QNa(t-\Delta t) + \Delta QNa \\
QK(t) &=& QK(t-\Delta t) + \Delta QK  \\
QA(t) &=& QA(t-\Delta t) + \Delta QA  \\
QCa(t) &=& QCa(t-\Delta t) + \Delta QCa  \\
QMg(t) &=& QMg(t-\Delta t) + \Delta QMg
\end{eqnarray}

$\Delta H$ is a special case because $\Delta H$ adds to the only titratable proton buffer nHb*QHb, so that:
\setcounter{equation}{0}
\renewcommand{\theequation}{21\alph{equation}}
\begin{eqnarray}
nHb(t)*QHb &=& nHb(t-\Delta t)*QHb + \Delta H \\
nHb(t) &=& nHb(t-\Delta t)  + \Delta H/QHb \\
\end{eqnarray}

From which we can now compute the new cell pHi from eq 4 by solving for pH(t):

\begin{equation}
pH(t) = nHb(t)/a + pI
\end{equation}
The new intracellular $H^+$ concentration in molar units is:
\begin{equation}
CH(t) = 10^{-pH(t)}
\end{equation}

With the new $Qi(t)$, we need the new cell water volume, $Vw(t)$ in order to compute the new cell concentrations, $Ci(t) = Qi(t)/Vw(t)$.  The water flux across the RBC membrane, $Fw$, is driven by the osmotic gradient across the RBC membrane (eqs 5 and 6): 
\setcounter{equation}{0}
\renewcommand{\theequation}{22\alph{equation}}

\begin{equation}
Fw(t) = Pw*(\COs(t) – \MOs(t-\Delta t)) 
\end{equation}
$\COs(t)$ can be computed from the altered osmotic load resulting from the $\Delta Qi$ changes during $\Delta t$ operating on the cell volume at the start of the each iteration interval: 

\begin{equation}
COs(t) = \frac{QNa(t) + QK(t) + Q(A)t + QCa(t) + QMg(t)}{Vw(t-\Delta t)}  + (f_{Hb}*\CHb + \CX )(t-\Delta t)
\end{equation}


\setcounter{equation}{0}
\renewcommand{\theequation}{23\alph{equation}}

The new cell water volume, $Vw(t)$, and volume-associated variables, $RCV(t)$, $MCHC(t)$, $Density(t)$ and $Hct(t)$, can now be computed from: 
\begin{eqnarray}
\Delta Vw(t) &=& Fw(t)*\Delta t \\
Vw(t) &=& Vw(t-\Delta t) + \Delta Vw \\
RCV(t) &=& 1-Vw(t=0) + Vw(t)  \\
MCHC(t) &=& MCHC(t=0)/RCV  \\
Density(t) &=& ((MCHC(t=0)/100) + Vw(t))/RCV  \\
Hct(t) &=& Htc(t=0)*RCV  
\end{eqnarray}

With $Vw(t)$ we proceed to compute next the new (t) intracellular concentrations of Na, K, A, H, Ca, Hb, and X:

\setcounter{equation}{0}
\renewcommand{\theequation}{24\alph{equation}}

\begin{eqnarray}
\CNa(t) &=& QNa(t)/Vw(t) \\
\CK(t) &=& QK(t)/Vw(t) \\
\CA(t) &=& QA(t)/Vw(t)  \\
\CCa(t) &=& QCa(t)/Vw(t)  \\
\CMg(t) &=& QMg(t)/Vw(t)  \\
\CHb(t) &=& QHb/Vw(t) \\
\CX^{-}(t) &=& QX^{-} /Vw(t)  
\end{eqnarray}

The new osmotic coefficient of Hb, $f_{Hb}(t)$, can now be calculated from eq 7 and the new $\CHb(t)$:

\setcounter{equation}{24}
\renewcommand{\theequation}{\arabic{equation}}

\begin{equation}
f_{Hb}(t) = 1 + b\times\CHb(t) + c\times CHb(t)^2
\end{equation}


\end{document}
